\documentclass{article}
\title{El lenguaje, un bien valioso en el capitalismo tardío}
\author{Abraham Calderón}
\date{Febrero, 2023}
\usepackage[utf8]{inputenc}
\usepackage{multicol}

\begin{document}
    \maketitle
    \begin{multicols}{2}
    \paragraph{}
    \columnbreak
      
    \paragraph{}
    \begin{flushright}
        \columnbreak
        ``El lenguaje desempeña un papel cada vez más central en la economía de la modernidad tardía.''
        \linebreak
        Loan Pujolar
    \end{flushright}


    \end{multicols}

    \paragraph{}
    En los últimos años, se ha producido un aumento significativo en la importancia atribuida al lenguaje. Sin embargo, es importante aclarar que no me refiero al lenguaje simplemente como la capacidad humana de comunicarse, ni tampoco a una lengua específica en particular. Me refiero al lenguaje como “el estilo, modo de hablar y de escribir de cada persona en particular”\footnote{Real Academia Española.\emph{ Diccionario de la lengua española}.(Buenos Aires: Paidos Ibérica, 2001)}.En la actualidad, el trabajo lingüístico ha adquirido una mayor relevancia en la economía, especialmente en los procesos de producción de bienes y servicios. De hecho, se destaca su importancia en este último aspecto en particular\footnote{Pujolar, Loan. ``La mercantilización de las lenguas (commodification)". En Claves para entender el multilingüismo contemporáneo, coord. L. \& J. (Zaragoza: Universidad de Zaragoza/Prensas Universitarias de Zaragoza, 2020), 132.}. El crecimiento exponencial de las empresas prestadoras de servicios es uno de los factores que ha llevado a que en la actualidad el lenguaje sea considerado un bien esencial en la sociedad.

    \paragraph{}
    En las últimas décadas, el sector terciario de la economía, que abarca servicios como entretenimiento, atención al cliente, consultorías, gobierno, entre otros, ha experimentado un crecimiento significativo en comparación con otros sectores\footnote{Heller, Monica. ``The Commodification of Language". Annuel Review of Anthropology, 39 (2010), 101-114.}. Una prueba del creciente valor del sector de servicios en las economías industrializadas se puede ver en el hecho de que representan el 70\% del valor agregado en países como Dinamarca, Francia y Estados Unidos\footnote{Ventura, V., Acosta, J., Durán, J., Kuwayama M. y Mattos J. 2003. \emph{Globalización y servicios: cambios estructurales en el comercio internacional}. (Santiago de Chile: CEPAL, 2003), 14.}. Además, en todo el mundo los servicios de streaming y contenido por internet han experimentado un crecimiento exponencial, impulsado por el desarrollo tecnológico y la globalización. Este aumento en la demanda de estos servicios ha generado un aumento en la demanda de profesionales con habilidades lingüísticas desarrolladas para brindar dichos servicios.
    \paragraph{}
    La creciente valoración del lenguaje no solo tiene un impacto en el ámbito laboral, sino también en la educación y en las relaciones interpersonales. El sociólogo francés, Pierre Bourdieu, señala, que el lenguaje se ha vuelto un bien simbólico, un bien con valor social, que puede influir en titulaciones académicas, pericia profesional e incluso en la percepción del aspecto físico\footnote{Bourdieu, Pierre. 1977. ``The economics of linguistic exchanges”. En Soc. Sci. Inf. 16(6). (1977), 649}. Heller coincide con este argumento, añadiendo que: ``El modo de hablar y escribir, puede ser la base para la valoración de uno como escolar, empleado o potencial pareja.”\footnote{Heller. ``The Commodification of Language", 103.}.
    \paragraph{}
    La búsqueda de eficiencia y competitividad para mejorar la economía y la sociedad ha llevado a que los avances tecnológicos impulsados por el capitalismo hayan sustituido la mayoría de los trabajos manuales en la industria y la manufactura por máquinas. Como resultado, las habilidades lingüísticas se han vuelto más esenciales en muchos trabajos, ya que los trabajadores deben tener habilidades de comunicación para desempeñarse efectivamente en el entorno laboral actual.\footnote{Pujolar. ``La mercantilización de las lenguas (commodification)", 136.}. Bajo esta misma lógica, las personas comienzan a percibir el lenguaje como una ventaja comparativa que les permite destacarse y diferenciarse de otros\footnote{Cameron, D. ``Estilizar al trabajador: género y mercantilización del lenguaje en la economía globalizada de servicios”. Traducción de Styling the worker: Gender and the commodification of language in \emph{the globalized service economy}. (Journal of Sociolinguistics,2000) 4/3, 324.}. Como resultado, muchas personas buscan perfeccionar y estilizar su uso del lenguaje para cumplir con sus objetivos personales. Para satisfacer esta necesidad, algunas empresas han comenzado a comercializar servicios destinados a mejorar el uso del lenguaje, dando lugar a lo que se conoce como la ``Mercantilización del Lenguaje"\footnote{Pujolar. ``La mercantilización de las lenguas (commodification)", 131.}. 

    \paragraph{}
    Aunque las empresas pueden cubrir esta necesidad, ¿qué sucede con aquellos que no pueden permitírselo? ¿Estarán en desventaja? Muchos gobiernos de países no industrializados no han logrado comprender los cambios en los mercados de servicios y la creciente demanda por trabajadores con habilidades lingüísticas avanzadas. En su lugar, algunos gobiernos piensan que la enseñanza del lenguaje como materia escolar obligatoria es suficiente, lo que resulta en un sistema educativo que no proporciona las habilidades prácticas necesarias para utilizar el lenguaje en situaciones sociales\footnote{Lorenzatti, María del Carmen. ``Concepciones y sentidos cotidianos acerca de la lectura y escritura de jóvenes y adultos".PAIVA. Aprendizados ao longo da vida: sujeitos, políticas e processos educativos. (Rio de Janeiro: EDUERJ, 2019), 109.}

    \paragraph{}
    En resumen, el lenguaje ha adquirido una mayor importancia debido al auge de las empresas de servicios. Este desarrollo ha sido impulsado por el capitalismo y la tecnología, lo que ha llevado a que el lenguaje sea considerado una ventaja comparativa en el mercado laboral. Para obtener esta ventaja, los individuos recurren a empresas que ofrecen servicios de enseñanza del lenguaje, ya que el sistema educativo público a menudo no proporciona las habilidades lingüísticas prácticas necesarias para competir en el mercado laboral actual.

\end{document}