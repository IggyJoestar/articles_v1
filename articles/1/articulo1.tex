\documentclass{article}
\title{El lenguaje, un bien valioso en el capitalismo tardío}
\author{Abraham Calderón}
\date{Octubre, 2022}
\usepackage[utf8]{inputenc}
\usepackage{multicol}

\begin{document}
    \maketitle
    \begin{multicols}{2}
    \paragraph{}
    \columnbreak
      
    \paragraph{}
    \begin{flushright}
        \columnbreak
        ``El lenguaje desempeña un papel cada vez más central en la economía de la modernidad tardía.''
        \linebreak
        Loan Pujolar
    \end{flushright}


    \end{multicols}

    \paragraph{}
    En los últimos años la valoración del lenguaje se ha incrementado significativamente. Pero no me refiero al lenguaje como la facultad del ser humano para expresarse, mucho menos como lengua. Me refiero al lenguaje como “el estilo, modo de hablar y escribir de cada persona en particular”\footnote{Real Academia Española.\emph{ Diccionario de la lengua española}.(Buenos Aires: Paidos Ibérica, 2001)}.Actualmente, el trabajo lingüístico ha tomado mayor relevancia en la economía, en los procesos de producción de bienes y servicios, destacando en este último\footnote{Pujolar, Loan. ``La mercantilización de las lenguas (commodification)". En Claves para entender el multilingüismo contemporáneo, coord. L. \& J. (Zaragoza: Universidad de Zaragoza/Prensas Universitarias de Zaragoza, 2020), 132.}. El crecimiento exponencial de las empresas prestadoras de servicios es el causante que en estos días se perciba al lenguaje como un bien superior en la sociedad.

    \paragraph{}
    En el sector terciario de la economía, sector relacionado a los servicios, conformado por el entretenimiento, atención al cliente, consultorías, gobierno entre otros; es el que más ha crecido en las últimas décadas\footnote{Heller, Monica. ``The Commodification of Language". Annuel Review of Anthropology, 39 (2010), 101-114.}. Los servicios son el 70\% del valor agregado en países industrializados como Dinamarca, Francia y Estados Unidos\footnote{Ventura, V., Acosta, J., Durán, J., Kuwayama M. y Mattos J. 2003. \emph{Globalización y servicios: cambios estructurales en el comercio internacional}. (Santiago de Chile: CEPAL, 2003), 14.}. Además, mundialmente los servicios de streaming e internet, han crecido exponencialmente gracias al desarrollo tecnológico y la globalización. Al crecer los servicios crece la demanda de gente que los brinda, generalmente personas con un buen uso del lenguaje.

    \paragraph{}
    Esta nueva valorización del lenguaje no solo afecta al campo laboral, también afecta a la escolaridad y a las relaciones interpersonales. El sociólogo francés, Pierre Bourdieu, señala, que el lenguaje se ha vuelto un bien simbólico, un bien con valor social, que puede influir en titulaciones académicas, pericia profesional e incluso en la percepción del aspecto físico\footnote{Bourdieu, Pierre. 1977. ``The economics of linguistic exchanges”. En Soc. Sci. Inf. 16(6). (1977), 649}. Heller coincide con este argumento, añadiendo que: ``El modo de hablar y escribir, puede ser la base para la valoración de uno como escolar, empleado o potencial pareja.”\footnote{Heller. ``The Commodification of Language", 103.}.
    \paragraph{}
    Los avances tecnológicos del capitalismo han provocado que la mayoría de los trabajos industriales y de manufactura sean realizados por maquinas. Por tanto, ya no es tan necesario un trabajador manual sino uno con buen lenguaje, siguiendo la lógica capitalista de buscar eficiencia y competitividad para la mejora de la población\footnote{Pujolar. ``La mercantilización de las lenguas (commodification)", 136.}. Bajo la misma lógica los individuos comienzan a percibir al lenguaje como una ventaja comparativa, que hace destacar a uno sobre otros\footnote{Cameron, D. ``Estilizar al trabajador: género y mercantilización del lenguaje en la economía globalizada de servicios”. Traducción de Styling the worker: Gender and the commodification of language in \emph{the globalized service economy}. (Journal of Sociolinguistics,2000) 4/3, 324.}. Por esa razón, los individuos buscan perfeccionarlo y estilizarlo acorde a sus intereses. Para satisfacer esta necesidad, la enseñanza de un mejor desempeño en el lenguaje, algunas firmas comenzaron a comercializar con el lenguaje creando la ``Mercantilización del Lenguaje”\footnote{Pujolar. ``La mercantilización de las lenguas (commodification)", 131.}. 

    \paragraph{}
    Las empresas cubren esta necesidad, pero ¿qué pasa con los que no puedan adquirirla? ¿estarían en una desventaja? Muchos gobiernos de países no industrializados no han logrado percibir los cambios en los mercados de servicios, y la creciente demanda por trabajadores con un buen uso del lenguaje. Ellos creen que solo con enseñar el uso de la lengua solo como escolaridad obligatoria es suficiente, cuando esto solo genera un sistema educativo que no ofrece técnicas para emplearlo en prácticas sociales.\footnote{Lorenzatti, María del Carmen. ``Concepciones y sentidos cotidianos acerca de la lectura y escritura de jóvenes y adultos".PAIVA. Aprendizados ao longo da vida: sujeitos, políticas e processos educativos. (Rio de Janeiro: EDUERJ, 2019), 109.}

    \paragraph{}
    En resumen, actualmente el lenguaje tiene mayor relevancia debido al desarrollo de las empresas prestadoras de servicios. Asimismo, el desarrollo de estas empresas es gracias al capitalismo y la tecnología, que al mismo tiempo posiciona al lenguaje como una ventaja comparativa. Para conseguir esta ventaja los individuos acuden a empresas que comercializan la enseñanza del lenguaje, ya que no se puede contar con lo aprendido en las escuelas públicas.

\end{document}